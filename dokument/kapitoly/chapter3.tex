\chapter{Related methodology and specific requirements for forensic audit}
\komentar{
tato kapitola by asi mela obsahovat oduvodnene pozadavky na vlastni metodiku, pouzitelnou ve FA s ohledem na ostatni pristupy a metodiky
}

%\komentar{jaky je rozdil mezi metodikou, kterou uz mam vytvorenou v arisu a touto vlastni metodikou?-> je to totez}

\komentar{
\section{Related methodology}
}\subsection{Project management in general}

Project management in general can be used in forensic audit when we see forensic audit itself as a project. The aim of the project is to resolve a given situation. 
There are multiple techniques for managing project activities in order to achieve specific goals and meet specific success criteria. One of the commonly used techniques is a network analysis. This technique is based on the fact that typically all projects can be broken down into separate activities and precedence relationships. \komentar{http://people.brunel.ac.uk/~mastjjb/jeb/or/netanal.html} The precedence relationships describe the fix order in which the activities may be performed. The aim is then to plan the activities so that the success criteria could be met. 

Essentially all the activities needed for the project to be finished are listed and then the time for finishing each one of them is evaluated. After that all the limiting conditions on the order of the activities is described. It means to state which activities must be completed before this one starts. Subsequently we get a list of immediate precedence relationships between activities. Sometimes a sequence of activities needs to be repeated in several iterations. Then we need to guess \komentar{odhadnout?} the number of iterations.

Once we have the activity list and precedence relationships we can compile a plan how to proceed from the start to a successful completion of the task. This can be done by creating a network diagram. 


\komentar{\subsection{ Obecne zpracovani objednavky (zde je objednavkou FA projekt)}
}\dotaz{Jak je tenhle bod myslen? Nejak nerozumim...}\sediva{\blindtext}

\subsection{Six Sigma}\komentar{http://www.sixsigma-iq.cz/COJESIXSIGMA.aspx}
Six sigma is an example of strategy management which is used in several branches of industry. It is a complex management method or rather management philosophy focused on continuous improving of efficiency. The main principle is based on improving the quality of the output by removing causes of defects. One of Six Sigma methodologies is DMAIC which is an abbreviation of following key words: Define, Measure, Analyze, Improve, Control.

\komentar{\section{Specificke pozadavky pro FA}

\komentar{Fraud and Forensic accounting in digital environment - zdroj,\\ data mining for fraud, advanced statistical methods}


}\sediva{\blindtext}
\komentar{\subsection{  Zohledneni rizik}
}\sediva{\blindtext}
\komentar{\subsection{ Zohledneni ruznych dat}
}\sediva{\blindtext}
\komentar{\subsection{Zohledneni ruznych zpusobu zpracovani a intepretace dat}
}\sediva{\blindtext}



\komentar{
kde je potreba volnost/ flexibilita a kde standardizovanost?\\

}
