\chapter*{Introduction} \label{Introduction}
% !TeX spellcheck = en_US
%uvedes problem, popises ho, proc je to problem, proc to resime, v cem nam to pomuze, kdyz ho vyresime; 
%a zminis, ze tato prace se snazi tento problem resit (ale uz ne moc jak).


%This bachelor project is concerned with the usage of software solutions in the field of forensic audit. 

%he basic definition of forensic audit is an assessment of a company's or individuals financial data or utilization as evidence in court. This procedure can be used with a specific end goal to indict for extortion, financial issues or embezzlement. 
%


%Forensic audit concentrates on the investigative procedure and its diverse stages and are represented by instinctive methodologies. This project proposes a dynamic model of the digital forensic model in the light of a new flow-based detailed system. It is demonstrated in samples that the technique can consistently indicate the forensic process in different stages and crosswise over portions. It likewise gives more correct depiction where things like data and confirmation are isolated into diverse streams of flow.


%\paragraph{Project Rationale}
There are two main areas in this project. At first the field of forensic audit and its frequently recurring processes are introduced. Secondly an information system for support of forensic audit is discussed and designed. The result of this project is a guide for a programmer to help easily implement a system usable in the field of forensic audit.


In the first part of this bachelor project we get to know the whole branch of forensic audit together with some of the issues that are commonly appearing while conducting the job of a forensic auditor. Next we propose several methods of forensic audit and we try to analyze whether a computer support is appropriate to be considered. Examples of available software solutions for selected cases are also offered. With regard to introduced methodology, we describe the requirements on our own system that could be helpful in the execution of forensic audit. In the following section a design of the new application is systematically described. UML diagrams are used to clarify the system design. A discussion about technologies to be used for implementation is provided in the end of this project. Final chapter is dedicated to a conclusion. 



\sediva{ \blindtext}
\sediva{ \blindtext}





















