\chapter{Software requirements specification}\label{SRS}


The purpose of this software requirements specification is to describe all the information needed to design and later implement anticipated software product. 

This bachelor degree project aims to provide an information system for support of forensic audit. \dotaz{Who is the audience of this text???} \dotaz{Jaky je celkovy prinos prace???????}




\subsection{User characteristics}
Users of this system are of several kinds. First of all it can be forensic auditors attempting to resolve an assigned case. They can use this software to observe the sequence of recorded events as they happened in time. This way the process of forensic audit can be supported. 

\komentar{\begin{itemize}
\item forenzni auditor
\item ostatni clenove tymu pracujici ho na forenznim auditu
\item manazer na sledovani vyvoje procesu FA
\item zadavatel na totez
\item pri prezentaci vysledku muze byt pouzit jako nastoj pro jednoduchou demonstraci sledu udalosti zadavateli
}


\section{Introduction}
\subsection{Purpose}

The purpose of this chapter is to present a detailed description of an information system that will support the process of forensic audit. It will explain the purpose and features of the system, the interfaces of the system, what the system will do, the constraints under which it must operate and how the system will react to external stimuli.

\subsection{Scope of Project}
This information system will serve as a tool that helps a team of forensic auditors to investigate a given case. The system will be designed to visually represent discovered facts and will provide a background for understanding and resolving the case. 

More specifically this system will provide visualization of all known events that happened during the investigated time period. The user of the system will be able to replay the actions and see how the situation developed in time in an animation. The animation can be used afterwards in presentation of results to the ordering party. 

\subesection{Glossary}
\begin{table}[]
\centering
\caption{My caption}
\label{my-label}
\begin{tabular}{ll}
\textbf{Term} & \textbf{Definition} \\
ISfSoFA       &Information System for Support of Forensic Audit\\
              &                     \\
              &                    
\end{tabular}
\end{table}

\subsection{Overview of Document}

\section {Overall Description}

\subsection{System Environment}











